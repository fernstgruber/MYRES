%%%%%%%%%%%%%%%%%%%%%%%%%%%%%%%%%%%%%%%%%
% Twenty Seconds Resume/CV
% LaTeX Template
% Version 1.1 (8/1/17)
%
% This template has been downloaded from:
% http://www.LaTeXTemplates.com
%
% Original author:
% Carmine Spagnuolo (cspagnuolo@unisa.it) with major modifications by 
% Vel (vel@LaTeXTemplates.com)
%
% License:
% The MIT License (see included LICENSE file)
%
%%%%%%%%%%%%%%%%%%%%%%%%%%%%%%%%%%%%%%%%%

%----------------------------------------------------------------------------------------
%	PACKAGES AND OTHER DOCUMENT CONFIGURATIONS
%----------------------------------------------------------------------------------------

\documentclass[letterpaper]{twentysecondcv} % a4paper for A4

%----------------------------------------------------------------------------------------
%	 PERSONAL INFORMATION
%----------------------------------------------------------------------------------------

% If you don't need one or more of the below, just remove the content leaving the command, e.g. \cvnumberphone{}

\profilepic{Fauser_crop3.jpeg} % Profile picture

\cvname{Teresa Fauser} % Your name
\cvjobtitle{} % Job title/career

\cvdate{18. September 1981} % Date of birth
\cvaddress{Anna-Stainer-Knittel-Weg 3/5/4\newline
6020 Innsbruck\newline
\"{O}sterreich} % Short address/location, use \newline if more than 1 line is required
\cvnumberphone{+43 650 8373729} % Phone number
\cvsite{} % Personal website
\cvmail{t.fauser@gmx.at} % Email address

%----------------------------------------------------------------------------------------

\begin{document}

%----------------------------------------------------------------------------------------
%	 ABOUT ME
%----------------------------------------------------------------------------------------

\aboutme{Ich arbeite gerne \textbf{mit} und \textbf{f\"ur} Menschen, Kommunikation ist mir ein gro{\ss}es Anliegen. \vspace{0.5em} \newline Weiters zeichnet mich aus: \newline $\bullet$ interkulturelle Sensibilit\"at \newline $\bullet$ hohe soziale Kompetenz \newline $\bullet$ Flexibilit\"at \newline $\bullet$ Begeisterungsf\"ahigkeit \newline $\bullet$ Verantwortungsbewusstsein} % To have no About Me section, just remove all the text and leave \aboutme{}
%----------------------------------------------------------------------------------------
%	 SKILLS
%----------------------------------------------------------------------------------------

% Skill bar section, each skill must have a value between 0 an 6 (float)
\skills{{Niederl\"andisch, Russisch/1.5},{Franz\"osisch/3},{Englisch/4.8},{Spanisch/5.8},{MS Office/4.5},{Kommunikation/4.5}}

%------------------------------------------------



%----------------------------------------------------------------------------------------

\makeprofile % Print the sidebar

%----------------------------------------------------------------------------------------
%	 EXPERIENCE
%----------------------------------------------------------------------------------------

\section{Berufserfahrung}

\vspace{1em}
\begin{twenty} % Environment for a list with descriptions
	\twentyitem{2016-}{Tiroler Soziale Dienste GmbH}{Innsbruck}{\vspace{-1em}\flushleft{$\bullet$ Inhaltliche und organisatorische Koordination im Bereich unbegleitete minderj\"ahrige Fl\"uchtlinge  \newline $\bullet$ Vernetzung mit Systempartnern  (Kinder- und
Jugendhilfe, SOS Kinderdorf, Rotes Kreuz etc.) \newline  $\bullet$ Konzeptarbeit  \newline $\bullet$ Mitwirkung am Organisationsentwicklungsprozess sowie im Qualit\"atsmanagement}} 
		\twentyitem{2013}{Voluntariat an der Unidad Educativa Ecuatoriana Austriaca in Pedro Carbo (Bildungskarenz)}{Ecuador}{\vspace{-1em}\flushleft{$\bullet$ p\"adagogische Unterst\"utzung im Kindergarten \newline $\bullet$ Englischunterricht in der Volks- und Mittelschule \newline $\bullet$ Englischnachhilfe f\"ur lernschwache Volkssch\"ulerInnen}}
	\twentyitem{2011-2015}{Fonds Soziales Wien \newline
	Grundversorgung f\"ur hilfs- und schutzbed\"urftige Fremde}{Wien}{\vspace{-1em}\flushleft{$\bullet$ Leistungszuerkennung \newline $\bullet$ Kontakt mit Vertragspartnern \newline $\bullet$ Mitwirkung an der Erstellung von  Leitf\"aden\newline $\bullet$ Einschulung neuer KollegInnen}}
	\twentyitem{2008-2010}{Botschaft von Peru in \"Osterreich}{Wien}{\vspace{-1em}\flushleft{$\bullet$ administrative T\"atigkeiten \newline $\bullet$ \"Ubersetzungen \newline $\bullet$ Organisation kultureller Events}}
	%\twentyitem{<dates>}{<title>}{<location>}{<description>}
\end{twenty}

%----------------------------------------------------------------------------------------
%	 EDUCATION
%----------------------------------------------------------------------------------------

\section{Ausbildung}

\vspace{1em}
\begin{twenty} % Environment for a list with descriptions
	\twentyitem{2000-2008}{\"Ubersetzerstudium}{Universit\"at Wien}{\vspace{-1em}\flushleft{Arbeitssprachen: Deutsch - Spanisch }}
	\twentyitem{2001-2007}{Geschichtestudium}{Universit\"at Wien}{\vspace{-1em}\flushleft{Schwerpunkt: Lateinamerikanische Geschichte }}
	\twentyitem{1992-2000}{Neusprachliches Bundesgymnasium}{Neunkirchen/N\"O}{}
	
	\twentyitem{1988-1992}{Volkschule}{Schwarzau/N\"O}{}
	%\twentyitem{<dates>}{<title>}{<location>}{<description>}
\end{twenty}


\section{Auslandsaufenthalte}

\vspace{1em}
\begin{twentyshort} % Environment for a short list with no descriptions
\twentyitemshort{Dezember 2013-Mai 2014}{\hspace*{1em}Reisen durch S\"udamerika  und die Antarktis}	
	\twentyitemshort{September 2013-November 2013}{\hspace*{1em}Auslandsaufenthalt in Pedro Carbo, Ecuador}

	\twentyitemshort{Mai-Juli 2007}{\hspace*{1em}Diplomarbeitsrecherchen in Venezuela}
	\twentyitemshort{November 2005}{\hspace*{1em}Studien-Exkursion nach Venezuela}
	\twentyitemshort{Sommersemester 2004 }{\hspace*{1em}Studium an der Universidad de Granada, Spanien}
	\twentyitemshort{August 1999}{\hspace*{1em}Sprachkurs in Malaga, Spanien}
	\twentyitemshort{Juli 1997}{\hspace*{1em}Sprachkurs in Hastings, England}
	\twentyitemshort{September 1996}{\hspace*{1em}Sprachkurs in Hastings, England}
	%\twentyitemshort{<dates>}{<title/description>}
\end{twentyshort}



\
%----------------------------------------------------------------------------------------
%	 INTERESTS
%----------------------------------------------------------------------------------------

\section{Interessen}

\vspace{1em}
\begin{twentyshort} % Environment for a short list with no descriptions
	\twentyitemshort{Interesse an fremden L\"andern und Kulturen}{\hspace*{0em}}
	\twentyitemshort{Reisen}{\hspace*{0em}}
	\twentyitemshort{Laufen}{\hspace*{0em}}
	\twentyitemshort{Wandern}{\hspace*{0em}}
	\twentyitemshort{G\"artnern im Gemeinschaftsgarten}{\hspace*{0em}}
	%\twentyitemshort{<dates>}{<title/description>}
\end{twentyshort}
%----------------------------------------------------------------------------------------
%	 SECOND PAGE EXAMPLE
%----------------------------------------------------------------------------------------

%\newpage % Start a new page

%\makeprofile % Print the sidebar

%\section{Other information}

%\subsection{Review}

%Alice approaches Wonderland as an anthropologist, but maintains a strong sense of noblesse oblige that comes with her class status. She has confidence in her social position, education, and the Victorian virtue of good manners. Alice has a feeling of entitlement, particularly when comparing herself to Mabel, whom she declares has a ``poky little house," and no toys. Additionally, she flaunts her limited information base with anyone who will listen and becomes increasingly obsessed with the importance of good manners as she deals with the rude creatures of Wonderland. Alice maintains a superior attitude and behaves with solicitous indulgence toward those she believes are less privileged.

%\section{Other information}

%\subsection{Review}

%Alice approaches Wonderland as an anthropologist, but maintains a strong sense of noblesse oblige that comes with her class status. She has confidence in her social position, education, and the Victorian virtue of good manners. Alice has a feeling of entitlement, particularly when comparing herself to Mabel, whom she declares has a ``poky little house," and no toys. Additionally, she flaunts her limited information base with anyone who will listen and becomes increasingly obsessed with the importance of good manners as she deals with the rude creatures of Wonderland. Alice maintains a superior attitude and behaves with solicitous indulgence toward those she believes are less privileged.

%----------------------------------------------------------------------------------------

\end{document} 
