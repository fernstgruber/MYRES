%%%%%%%%%%%%%%%%%%%%%%%%%%%%%%%%%%%%%%%%%
% Twenty Seconds Resume/CV
% LaTeX Template
% Version 1.1 (8/1/17)
%
% This template has been downloaded from:
% http://www.LaTeXTemplates.com
%
% Original author:
% Carmine Spagnuolo (cspagnuolo@unisa.it) with major modifications by 
% Vel (vel@LaTeXTemplates.com)
%
% License:
% The MIT License (see included LICENSE file)
%
%%%%%%%%%%%%%%%%%%%%%%%%%%%%%%%%%%%%%%%%%

%----------------------------------------------------------------------------------------
%	PACKAGES AND OTHER DOCUMENT CONFIGURATIONS
%----------------------------------------------------------------------------------------

\documentclass[a4paper]{twentysecondcv} % a4paper for A4
\usepackage{ClearSans}
\usepackage[authoryear]{natbib}


%----------------------------------------------------------------------------------------
%	 PERSONAL INFORMATION
%----------------------------------------------------------------------------------------

% If you don't need one or more of the below, just remove the content leaving the command, e.g. \cvnumberphone{}

\profilepic{Foto_FabianGruber.jpeg} % Profile picture

\cvname{Fabian Gruber} % Your name
\cvjobtitle{Wiss. Projektmitarbeiter} % Job title/career

\cvdate{22 Juni 1982} % Date of birth
\cvaddress{Anna-Stainer-Knittel-Weg 3/5/4\newline
6020 Innsbruck\newline
Austria} % Short address/location, use \newline if more than 1 line is required
\cvnumberphone{+43 650 258 75 21} % Phone number
\cvsite{} % Personal website
\cvmail{Fabian.Gruber@uibk.ac.at} % Email address


%----------------------------------------------------------------------------------------

\begin{document}

%----------------------------------------------------------------------------------------
%	 ABOUT ME
%----------------------------------------------------------------------------------------

\aboutme{} % To have no About Me section, just remove all the text and leave \aboutme{}


%----------------------------------------------------------------------------------------
%	 SKILLS
%----------------------------------------------------------------------------------------

% Skill bar section, each skill must have a value between 0 an 6 (float)

\skills{{Spanish/1.5},{English/5.5},{German/6.0},{\LaTeX \,\,\& \textsc{Bib}\TeX /3.8},{GIMP and Incscape/3.5},
{Adobe Illustrator/5.4},{Arc GIS/4.1},{SAGA GIS/5.1},{GRASS GIS/5.5},{Python programming language/4.2},{R programming language/5.2}}


%----------------------------------------------------------------------------------------

\makeprofile % Print the sidebar



%----------------------------------------------------------------------------------------
%	 INTERESTS
%----------------------------------------------------------------------------------------

\section{Fachgebiete}

\emph{Bodenkunde:} Bodenfunktionsbewertung, Digital soil mapping, Bodensystematik\\[0.3em]
\emph{Geologie:} Bodenausgangsmaterialkartierung, Geohydrologie\\[0.3em]
%\emph{Fernerkundung:} monitoring of slope deformations, terrestrial laser scanning\\[0.3em]
\emph{Geoinformatik:} Open Source GIS (GRASS, SAGA, QGIS, GDAL/OGR)\\[0.3em]
\emph{Datenanalyse:} Deskriptive Statistik und Datenvisualisierung mit R\\[0.3em]
\emph{Modellierung:} statistische Modellierung (mit R) mit machine learning Ans\"atzen\\
\hspace*{6em}           Naturgefahren-Modellierung mit BREACH, FLO-2D, RAMMS, DAN-3D\\

%----------------------------------------------------------------------------------------
%	 EXPERIENCE AND TEACHING
%----------------------------------------------------------------------------------------

\section{Berufserfahrung}

\begin{twentymid}
	\twentyitemmid{2013 -- 2018}{Wiss. Mitarbeiter}{Institut f\"ur Geographie, Universit\"at Innsbruck}
	\twentyitemmid{2016 -- 2017}{Lektor}{LV \"{U}bungen zur Statistik mit R, Universit\"at Innsbruck}
	\twentyitemmid{2011 -- 2013}{Wiss. Mitarbeiter}{Institut f\"ur Angewandte Geologie, BOKU, Wien}
	\twentyitemmid{2009 -- 2010}{Tutor}{LV Einf\"uhrung in GIS, BOKU Wien}
	\twentyitemmid{2009 -- 2010}{Projektmitarbeiter}{Institut f\"ur Angewandte Geologie, BOKU Wien}
\end{twentymid}


\vspace*{1em}












%----------------------------------------------------------------------------------------
%	 EDUCATION
%----------------------------------------------------------------------------------------

\section{Ausbildung}

\begin{twentymid} % Environment for a list with descriptions
	\twentyitemmid{2013 -- }{PhD in Geography}{University of Innsbruck, Austria}
	\twentyitemmid{2002 -- 2011}{Diplomstudium Kulturtechnik \& Wasserwirtschaft}{BOKU Wien}
	\twentyitemmid{2001 -- 2002}{Zivildienst}{Arbeitersamariterbund, Linz, Austria}
	\twentyitemmid{1991 -- 1993}{Gymnasium}{Linz International School, Austria}
	\twentyitemmid{1991 -- 1993}{Volksschule}{Linz, Austria}
	\twentyitemmid{1989 -- 1991}{Volksschule}{Pittsburgh, PA., USA}
\end{twentymid}

\vspace*{0.7em}



%----------------------------------------------------------------------------------------
%	 PUBLICATIONS
%----------------------------------------------------------------------------------------

\section{Ausgew\"ahlte Publikationen}


\let\oldsection=\section
\renewcommand{\section}[2]{}%
{\footnotesize
\nocite{*}
\bibliography{pub_FEG}
\bibliographystyle{thesis-harv}
}
\let\section=\oldsection

\vspace*{1.3em}

%----------------------------------------------------------------------------------------
%	 AWARDS
%----------------------------------------------------------------------------------------

%\section{Awards}

%\begin{twentyshort} % Environment for a short list with no descriptions
%	\twentyitemshort{1987}{All-Time Best Fantasy Novel.}
%	\twentyitemshort{1998}{All-Time Best Fantasy Novel before 1990.}
%	%\twentyitemshort{<dates>}{<title/description>}
%\end{twentyshort}



%----------------------------------------------------------------------------------------
%	 REFERENCES
%----------------------------------------------------------------------------------------

\section{Referenzen}

\begin{references} % Environment for a list with descriptions
	\refitem{Clemens Geitner}{+43 507 54037}{Institut f\"ur Geographie,}{clemens.geitner@uibk.ac.at}{Universit\"at Innsbruck}	
	\refitem{Martin Mergili}{+43 1 47654 87219}{Institut f\"ur Angewandte Geologie,}{martin.mergili@boku.ac.at}{Universtit\"at f\"ur Bodenkultur}
	\refitem{Martin Rutzinger}{+43 507 49480}{Institut f\"ur Interdisziplin\"are Gebirgsforschung,}{martin.rutzinger@oeaw.ac.at}{\"{O}sterr. Akademie der Wissenschaften}

\end{references}

\vspace*{0.7em}


%----------------------------------------------------------------------------------------
%	 OTHER INFORMATION
%----------------------------------------------------------------------------------------

\section{Interessen}

\emph{Reisen: } Mittelamerika, Zentral und S\"udostasien, Madagaskar, Antarktis\\
\emph{Gem\"useanbau: } Mitarbeit im Gemeinschaftsgarten Waldh\"uttl, Innsbruck\

\end{document} 
